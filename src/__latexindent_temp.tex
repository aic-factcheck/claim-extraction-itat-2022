\section{Conclusion}\label{sec:conclusion}

With this paper, we aimed to address the lack of a Czech dataset for automated fact-checking.
We have explored two ways of acquiring such data.

Firstly, we localize the \FEN dataset, using a document alignment between Czech and English \Wikipedia abstracts extracted from the \textit{interlingual links}.
We obtain and publish the \FCZ dataset of 127k machine-translated claims with evidence enclosed within the Czech \Wikipedia dump.
We then validate our alignment scheme and measure a 66\% precision using hand annotations over a 1\% sample of obtained data.
Therefore, we recommend the data for models less sensitive to noise and utilize it to train experimental DR models and recall estimation.

Secondly, we execute a series of human annotation runs with 163 students of journalism to acquire a novel dataset in Czech.
As opposed to similar annotations that extracted the claims and evidence from \Wikipedia~\cite{fever2018,norregaard2021danfever,aly2021feverous}, we annotate our dataset on top of a CTK corpus extracted from a news agency archive to explore this different relevant language form. 
We collect a raw dataset of 3,116 labeled claims, 57\% of which have at $\geq2$ independent cross-annotations.
From these, we calculate Krippendorff's alpha to be 56.42\%.
We proceed with manual and human-and-model-in-the-loop annotation cleaning to remove conflict and malformed annotations and arrive at a thoroughly cleaned \CTK dataset of 3,097 claims and their veracity annotations with evidence from the CTK corpus.
We release its version for NLI called \CTKNLI to maintain corpus trade secrecy.

Finally, we use our datasets to train baseline models for the full fact-checking pipeline composed of Document Retrieval and Natural Language Inference tasks.
% At five retrieved paragraphs in a Score Evidence setting, where at least one gold set of evidence must be covered by the DR predictions, our best performing models scored 33.89\% overall accuracy on \FCZ and 20.14\% on \CTK.
% In a No Score Evidence setting, where the gold evidence requirement is removed, we scored 55.40\% and 61.24\% on \FCZ and \CTK, respectively.
% We claim the results to be testifying to the viability of the task in our setting and encouraging for further research on our data.

\subsection{Future work}
\begin{itemize}
    \item The fact-checking pipeline is to also be augmented by the \textit{check-worthiness estimation}~\cite{nakov2021automated}, that is, a model that classifies which sentences of a given text in Czech are appropriate for the fact verification.
    We are currently working on models that detect claims within the Czech Twitter, and a strong predictor for this task would also strengthen our annotation scheme from Section~\ref{sec:annotation} that currently relies on hand-picked check-worthy documents.
    \item While the \SUP, \REF{} and \NEI{} classes  offer a finer classification w.r.t. evidence than binary \texttt{true}/\texttt{false}, it is a good convention of fact-checking services to use additional labels, s.a. \texttt{MISINTERPRETED}, that could be integrated into the common automated fact verification scheme if well formalised.
    \item The enclosed annotation schemes for this task like that from~\cite{fever2018} or Section~\ref{sec:annotation} do not necessarily produce organic claims capturing the real-world complexity of fact-checking -- for example, just the \FEN \train set contains hundreds of claims of form \enquote{X is a person.}.
    This does not have a trivial solution, but we suggest an integration of real-world sources of claims, s.a. Twitter, into the annotation scheme.
    \item While the \FEVER localization scheme from Section~\ref{fcz-method} yielded a rather noisy dataset, its size and document precision encourage
    deployment of a model-based cleaning scheme like that from~\cite{Jeatrakul} to further refine its results.
\end{itemize}