%%%%%%%%%%%%%%%%%%%%%%%%%%%%%%%%%%%%%%%%%%%%%%%%%%%%%%%%%%%%%%%%%%%%%
%%                                                                 %%
%% Please do not use \input{...} to include other tex files.       %%
%% Submit your LaTeX manuscript as one .tex document.              %%
%%                                                                 %%
%% All additional figures and files should be attached             %%
%% separately and not embedded in the \TeX\ document itself.       %%
%%                                                                 %%
%%%%%%%%%%%%%%%%%%%%%%%%%%%%%%%%%%%%%%%%%%%%%%%%%%%%%%%%%%%%%%%%%%%%%

%%\documentclass[referee,sn-basic]{sn-jnl}% referee option is meant for double line spacing

%%=======================================================%%
%% to print line numbers in the margin use lineno option %%
%%=======================================================%%

%%\documentclass[lineno,sn-basic]{sn-jnl}% Basic Springer Nature Reference Style/Chemistry Reference Style

%%======================================================%%
%% to compile with pdflatex/xelatex use pdflatex option %%
%%======================================================%%

%%\documentclass[pdflatex,sn-basic]{sn-jnl}% Basic Springer Nature Reference Style/Chemistry Reference Style

%%\documentclass[sn-basic]{sn-jnl}% Basic Springer Nature Reference Style/Chemistry Reference Style
\documentclass[sn-mathphys]{sn-jnl}% Math and Physical Sciences Reference Style
%%\documentclass[sn-aps]{sn-jnl}% American Physical Society (APS) Reference Style
%%\documentclass[sn-vancouver]{sn-jnl}% Vancouver Reference Style
%%\documentclass[sn-apa]{sn-jnl}% APA Reference Style
%%\documentclass[sn-chicago]{sn-jnl}% Chicago-based Humanities Reference Style
%%\documentclass[sn-standardnature]{sn-jnl}% Standard Nature Portfolio Reference Style
%%\documentclass[default]{sn-jnl}% Default
%%\documentclass[default,iicol]{sn-jnl}% Default with double column layout

%%%% Standard Packages
%%<additional latex packages if required can be included here>
%%%%

%%%%%=============================================================================%%%%
%%%%  Remarks: This template is provided to aid authors with the preparation
%%%%  of original research articles intended for submission to journals published 
%%%%  by Springer Nature. The guidance has been prepared in partnership with 
%%%%  production teams to conform to Springer Nature technical requirements. 
%%%%  Editorial and presentation requirements differ among journal portfolios and 
%%%%  research disciplines. You may find sections in this template are irrelevant 
%%%%  to your work and are empowered to omit any such section if allowed by the 
%%%%  journal you intend to submit to. The submission guidelines and policies 
%%%%  of the journal take precedence. A detailed User Manual is available in the 
%%%%  template package for technical guidance.
%%%%%=============================================================================%%%%

\jyear{2021}%

%% as per the requirement new theorem styles can be included as shown below
\theoremstyle{thmstyleone}%
\newtheorem{theorem}{Theorem}%  meant for continuous numbers
%%\newtheorem{theorem}{Theorem}[section]% meant for sectionwise numbers
%% optional argument [theorem] produces theorem numbering sequence instead of independent numbers for Proposition
\newtheorem{proposition}[theorem]{Proposition}% 
%%\newtheorem{proposition}{Proposition}% to get separate numbers for theorem and proposition etc.

\theoremstyle{thmstyletwo}%
\newtheorem{example}{Example}%
\newtheorem{remark}{Remark}%

\theoremstyle{thmstylethree}%
\newtheorem{definition}{Definition}%

\raggedbottom
%%\unnumbered% uncomment this for unnumbered level heads

\graphicspath{{fig/}}

\usepackage[utf8]{inputenc}
\usepackage{comment}
\usepackage{commath} % for \abs{}
\usepackage{csquotes}
\usepackage{hhline}
\usepackage{tablefootnote}
\usepackage{tabularx}
\usepackage{xcolor}
\usepackage{xspace}

\newcommand{\FEVER}{\textsc{FEVER}\xspace}
\newcommand{\FEVERNLI}{\textsc{FEVER-NLI}\xspace}
\newcommand{\Wikipedia}{\textsc{Wikipedia}\xspace}
\newcommand{\MediaWiki}{\textsc{MediaWiki}\xspace}
\newcommand{\FCZ}{\textsc{CsFEVER}\xspace}
\newcommand{\FCZNLI}{\textsc{CsFEVER-NLI}\xspace}
\newcommand{\FEN}{\textsc{EnFEVER}\xspace}
\newcommand{\FDAN}{\textsc{DanFEVER}\xspace}
\newcommand{\CTK}{\textsc{CTKFacts}\xspace}
\newcommand{\CTKNLI}{\textsc{CTKFactsNLI}\xspace}
\newcommand{\Anserini}{\textsc{Anserini}\xspace}
\newcommand{\DrQA}{\textsc{DrQA}\xspace}
\newcommand{\BERT}{\textsc{Bert}\xspace}
\newcommand{\RoBERTa}{\textsc{RoBERTa}\xspace}
\newcommand{\MBERT}{\textsc{M-Bert}\xspace}
\newcommand{\SMBERT}{\textsc{Sentence M-Bert}\xspace}
\newcommand{\ColBERT}{\textsc{ColBert}\xspace}
\newcommand{\SlavicBERT}{\textsc{SlavicBERT}\xspace}
\newcommand{\CZERT}{\textsc{Czert}\xspace}
\newcommand{\RobeCzech}{\textsc{RobeCzech}\xspace}
\newcommand{\XLM}{\textsc{XLM-RoBERTa}\xspace}
\newcommand{\FERNETC}{\textsc{FERNET-C5}\xspace}
\newcommand{\FERNETN}{\textsc{FERNET-News}\xspace}
\newcommand{\XLMSQUAD}{\textsc{XLM-RoBERTa @ SQuAD2}\xspace}
\newcommand{\XLMXNLI}{\textsc{XLM-RoBERTa @ XNLI}\xspace}

\newcommand{\train}{\textsf{train}\xspace}
\newcommand{\dev}{\textsf{dev}\xspace}
\newcommand{\test}{\textsf{test}\xspace}

\newcommand{\SUP}{\texttt{SUPPORTS}}
\newcommand{\REF}{\texttt{REFUTES}}
\newcommand{\NEI}{\texttt{NEI}}


\newcommand{\Tzero}{{$\textsf{T}_{\textsf{0}}$}}
\newcommand{\Tone}{{$\textsf{T}_{\textsf{1}}$}}
\newcommand{\ToneA}{{$\textsf{T}_{\textsf{1a}}$}}
\newcommand{\ToneB}{{$\textsf{T}_{\textsf{1b}}$}}
\newcommand{\Ttwo}{{$\textsf{T}_{\textsf{2}}$}}
\newcommand{\TtwoA}{{$\textsf{T}_{\textsf{2a}}$}}
\newcommand{\TtwoB}{{$\textsf{T}_{\textsf{2b}}$}}

\newcommand{\q}[1]{``#1''}
\newcommand{\qit}[1]{\textit{``#1''}}
\newcommand{\revise}[1]{{\color{blue}#1}}
\newcommand{\rewrite}[1]{{\color{red}#1}}
\newcommand{\red}[1]{{\color{red}#1}}
\newcommand{\revision}[1]{{\color{purple}#1}}
\newcommand{\jd}[1]{{\color{orange}\textbf{jd: }#1}}
\newcommand{\hu}[1]{{\color{blue}\textbf{hu: }#1}}
\newcommand{\mr}[1]{{\color{green}\textbf{mr: }#1}}
\newcommand{\todo}[1]{{\color{red}\colorbox{yellow}{\textbf{TODO: }}#1}}

\definecolor{delim}{RGB}{20,105,176}
\colorlet{punct}{red!60!black}
\colorlet{numb}{magenta!60!black}

\lstset{
  basicstyle=\ttfamily,
  columns=fullflexible,
  frame=single,
  breaklines=true,
  postbreak=\mbox{\textcolor{red}{$\hookrightarrow$}\space},
}

\lstdefinelanguage{json}{
    basicstyle=\normalfont\ttfamily,
    numbers=left,
    numberstyle=\scriptsize,
    stepnumber=1,
    numbersep=8pt,
    showstringspaces=false,
    breaklines=true,
    frame=lines,
    backgroundcolor=\color{white},
    literate=
     *{0}{{{\color{numb}0}}}{1}
      {1}{{{\color{numb}1}}}{1}
      {2}{{{\color{numb}2}}}{1}
      {3}{{{\color{numb}3}}}{1}
      {4}{{{\color{numb}4}}}{1}
      {5}{{{\color{numb}5}}}{1}
      {6}{{{\color{numb}6}}}{1}
      {7}{{{\color{numb}7}}}{1}
      {8}{{{\color{numb}8}}}{1}
      {9}{{{\color{numb}9}}}{1}
      {:}{{{\color{punct}{:}}}}{1}
      {,}{{{\color{punct}{,}}}}{1}
      {\{}{{{\color{delim}{\{}}}}{1}
      {\}}{{{\color{delim}{\}}}}}{1}
      {[}{{{\color{delim}{[}}}}{1}
      {]}{{{\color{delim}{]}}}}{1},
}

% Define where - description of equation
\usepackage{enumitem}
\SetLabelAlign{myright}{\hss\llap{$#1$}}
\newlist{where}{description}{1}
\setlist[where]{labelwidth=2cm, labelsep=1em, leftmargin=!, align=myright, font=\normalfont}

\begin{document}

% \title[\CTK: fact-checking dataset based on Czech news corpus]{\CTK: fact-checking dataset based on Czech news corpus}


\newcommand{\papertitle}{Extracting factual claims in Czech}
\title[\papertitle]{\revision{\papertitle}}

%%=============================================================%%
%% Prefix	-> \pfx{Dr}
%% GivenName	-> \fnm{Joergen W.}
%% Particle	-> \spfx{van der} -> surname prefix
%% FamilyName	-> \sur{Ploeg}
%% Suffix	-> \sfx{IV}
%% NatureName	-> \tanm{Poet Laureate} -> Title after name
%% Degrees	-> \dgr{MSc, PhD}
%% \author*[1,2]{\pfx{Dr} \fnm{Joergen W.} \spfx{van der} \sur{Ploeg} \sfx{IV} \tanm{Poet Laureate} 
%%                 \dgr{MSc, PhD}}\email{iauthor@gmail.com}
%%=============================================================%%

\author*[1]{\revision{\fnm{Herbert} \sur{Ullrich}}}\email{ullriher@fel.cvut.cz}
\equalcont{These authors contributed equally to this work.}

\author*[1]{\revision{\fnm{Jan} \sur{Drchal}}}\email{drchajan@fel.cvut.cz}
\equalcont{These authors contributed equally to this work.}


\affil*[1]{\orgdiv{Artificial Intelligence Center}, \orgname{Faculty of Electrical Engineering, Czech Technical University in Prague}, \orgaddress{\street{Charles Square 13}, \city{Prague~2}, \postcode{120 00}, \country{Czech Republic}}}

%%==================================%%
%% sample for unstructured abstract %%
%%==================================%%
\abstract{

\todo{Rewrite from scratch}

\revision{Datasets are available at \url{https://huggingface.co/datasets/ctu-aic/ctkfacts_nli} and \url{https://huggingface.co/datasets/ctu-aic/ctkfacts_nli}.}
}

% max 6 Keywords
\keywords{Automated Fact-Checking, Czech, Document Retrieval, Natural Language Inference, FEVER}

%%\pacs[JEL Classification]{D8, H51}

%%\pacs[MSC Classification]{35A01, 65L10, 65L12, 65L20, 65L70}

\maketitle

%!TEX ROOT=../itat2022.tex
\begin{comment}
  \section*{General TODO}
  \begin{itemize}
      \item {
        \jd{Do we have any data on how many articles were filtered during \Tzero?}
        \\
        \hu{This does not prevent slight inaccuracies, however, one could SELECT all the paragraphs in DB that were not used in any knowledge scope, group them by article, select max(candidate\_of) and calculate the number of NULL entries.}
      }
      % \item {
      %   \jd{Maybe we should call our dataset \textsc{CsFEVER} instead of \textsc{FEVER CS} (or most likely correctly \textsc{CzFEVER}) in accordance with the \textsc{DanFEVER} paper~\cite{norregaard2021danfever}. We can also write \FEN instead of "original" \FEN, similarly to the \textsc{DanFEVER} paper. \red{I have done this: use \texttt{\textbackslash FCZ}, \texttt{\textbackslash FEN}, and \texttt{\textbackslash CTK} commands.}}
      %   \\
      %   \hu{I agree with CsFEVER -- CzFEVER deceptively sounds more correct, but it is not (CZ is country code, but what we want to use are the NLP-standard ISO 639-1 language codes: \url{https://en.wikipedia.org/wiki/List_of_ISO_639-1_codes} -- similar to chinese="zh", china="cn")}
      % }
      \item \jd{"DANFEVER" paper says: "The claims focus on the same entity as the substring’s source document and may be supported by the text in the substring, but may also be refuted or unverifiable by the substring." -- compare to the \FEN paper -- I think that initially only supported claims are generated, only after then you get REFUTE and NEI by the mutations.}
      \\ \hu{True that.}
      \item \todo{Put sizes of both datasets to Abstract, Intro and Conclusion.}
      \item \todo{Motivate the "paragraph granularity" in Intro, also (now covered in \FCZ section).} 
  \end{itemize}
\end{comment}


\begin{comment}
TOFINALIZE: 
  2. Pragmatic (extra linguistic) knowledge and context is crucial. For
    example, imagine an example: "Country X did not invade country Y", Z said.
    Here the statement depends on the Z's position. Of course this is more
    a philosophical remark and I do not see how you can prevent anyone of
    misusing the approach, but maybe you could consider adding one or two
    sentences if you also agree this should be commented upon?
\end{comment}
\section{Introduction}\label{sec:intro}

In the current highly connected online society, the ever-growing information influx eases the spread of false or misleading news.
The omnipresence of fake news motivated formation of fact-checking organizations such as AFP Fact Check,\footnote{\url{https://factcheck.afp.com/}} International Fact-Checking Network,\footnote{\url{https://www.poynter.org/ifcn/}} PolitiFact,\footnote{\url{https://www.politifact.com/}} Poynter,\footnote{\url{https://www.poynter.org/}} Snopes,\footnote{\url{https://www.snopes.com/}} and many others.
At the same time, many tools for fake news detection and fact-checking are being developed: ClaimBuster~\cite{hassan2017claimbuster}, ClaimReview;\footnote{\url{https://www.claimreviewproject.com/}} or CrowdTangle\footnote{\url{https://www.crowdtangle.com/}} see~\cite{zeng2021fcsurvey} for more examples.
Many of these are based on machine learning technologies aimed at image recognition, speech to text, or Natural Language Processing (NLP).
This paper deals with the latter, focusing on automated fact-checking (hereinafter also referred to as \textit{fact verification}).

Automated fact verification is a complex NLP task~\cite{thorne2018automated} in which the veracity of a textual \textit{claim} gets evaluated with respect to a ground truth corpus.
The output of a fact-checking system gives a classification of the claim -- conventionally varying between \textit{supported}, \textit{refuted} and \textit{not enough information} available in corpus. 
For the \textit{supported} and \textit{refuted} outcomes it further supplies the \textit{evidence}, i.e., a list of documents that explain the verdict.
Fact-checking systems typically work in two stages~\cite{fever2018}. 
In the first stage, based on the input \textit{claim}, the Document Retrieval (DR) module selects the \textit{evidence}.
In the second stage, the Natural Language Inference module matches the \textit{evidence} with the \textit{claim} and provides the final verdict.    
Table~\ref{tab:fc_example} shows an example of data used to train the fact-checking systems of this type.

\begin{table}
  \begin{center}
  \begin{minipage}{0.83\textwidth}
  \caption{Truncated example from \CTK \train set.}\label{tab:fc_example}
  \begin{tabular}{p{\linewidth}}
  \toprule
  \textbf{Claim:} Spojené státy americké hraničí s Mexikem.\\ 
  \textbf{EN Translation:} \textit{The United States of America share borders with Mexico.}\\ 
  \midrule
  \textbf{Verdict:} \SUP\\ 
  \midrule
  \textbf{Evidence 1:} \q{Mexiko a USA sdílejí 3000 kilometrů dlouhou hranici, kterou ročně překročí tisíce Mexičanů v naději na lepší životní podmínky (\dots)}\\ 
   \textbf{EN:} \textit{Mexico and the U.S. share a 3,000-kilometre border, thousands of Mexicans cross each year in hopes of better living conditions (\dots)}\\\\
  \textbf{Evidence 2:} \q{Mexiko také nelibě nese, že Spojené státy stále budují na vzájemné, několik tisíc kilometrů dlouhé hranici zeď, která má zabránit fyzickému ilegálnímu přechodu Mexičanů do USA (\dots)}\\ 
   \textbf{EN:} \textit{Mexico is also uncomfortable with the fact that the United States is still building a wall on their mutual, several thousand-mile borders to prevent Mexicans from physically crossing illegally into the U.S. (\dots)}\\
  \botrule
  \end{tabular}
  \end{minipage}
  \end{center}
\end{table}

Current state-of-the-art methods applied to the domain of automated fact-checking are typically based on large-scale neural language models~\cite{fever2018b}, which are notoriously data-hungry. 
While there is a reasonable number of quality datasets available for high-profile world languages~\cite{zeng2021fcsurvey}, the situation for the low-resource languages is significantly less favorable.
Also, most available large-scale datasets are built on top of \Wikipedia~\cite{fever2018,aly2021feverous,schuster-etal-2021-vitaminc,sathe2020automated}. 
While encyclopedic corpora are convenient for dataset annotation, these are hardly the only eligible sources of the ground truth. 

We argue that corpora of verified news articles used as claim verification datasets are a relevant alternative to encyclopedic corpora.
Advantages are clear: the amount and detail of information covered by news reports are typically higher.
Furthermore, the news articles typically inform on recent events attracting public attention, which also inspire new fake or misleading claims spreading throughout the online space.

On the other hand, news articles address a more varied range of issues and have a more complex structure from the NLP perspective.
While encyclopedic texts are typically concise and focused on facts, the style of news articles can vary wildly between different documents or even within a single article.
For example, it is common that a report-style article is intertwined with quotations and informative summaries.
Also, claim validity might be obscured by complex temporal or personal relationships: a past quotation like \qit{Janet Reno will become a member of the Cabinet.} may or may not support the claim \qit{Janet Reno was the member of the Cabinet.}\footnote{\revision{And the veracity of such claim may be further nuanced by the affiliation and bias of the speaker.}}
This depends on, firstly, which date we verify the claim validity to, and secondly, who was or what was the competence of the quotation's author.
Note that similar problems are less likely in encyclopedia-based datasets like \FEVER\cite{fever2018}.


The contributions of this paper are as follows:
\begin{comment}
  TODO:
  In contribution (3.), I think the inter-annotator agreement should not really
be counted as  detailed analysis of the CTKFacts dataset, as it is, and should
be, an inherent and mandatory part of contribution (2.) CTKFacts.
\end{comment}
\begin{enumerate}
    \item \textbf{\FCZ:} We propose an experimental Czech localization of the large-scale \FEVER~\cite{fever2018} fact-checking dataset, utilizing the public \MediaWiki interlingual document alignment of \Wikipedia articles and a MT-based claim transduction.
    We publish our procedure to be used for other languages, and analyze its pitfalls.
    We denote the original English \FEVER as \FEN in the following sections to distinguish various language mutations.
    \item \textbf{CTKFacts:} we introduce a new Czech fact-checking dataset manually annotated on top of approximately two million Czech News Agency\footnote{\url{https://www.ctk.eu/}} news reports from 2000--2020.
    Inspired by \FEVER, we provide an updated and extended annotation methodology aimed at annotations of news corpora, and we also make available an open-source annotation platform.
    The \textit{claim generation} as well as \textit{claim labeling} is centered around limited  knowledge context (denoted \textit{dictionary} in~\cite{fever2018}), which is trivial to construct for hyperlinked textual corpora such as \Wikipedia.
    We present a novel approach based on document retrieval and clustering.
    The method automatically generates dictionaries, which are composed of both relevant and semantically diverse documents, and does not depend on any inter-document linking.
    \item We provide a detailed analysis of the \CTK dataset, including the \textit{inter-annotator agreement}, and \textit{spurious cue analysis}, where the latter detects annotation patterns possibly leading to overfitting of the NLP models.
    For comparison, we analyze the spurious cues of \FCZ as well.
    We construct an annotation cleaning scheme that involves both manual and semi-automated procedures, and we use it to refine the final version of the \CTK dataset.
    We also provide classification and discussion of common annotation errors for future improvements of the annotation methodology.
    \item We present baseline models for both DR and NLI stages as well as for the full fact-checking pipeline.
    \item We publicly release the \CTK dataset as well as the experimental \FCZ data, used source code and the baseline models\footnote{\url{https://github.com/aic-factcheck/csfever-and-ctkfacts-paper}. Note, that only the NLI part of the \CTK  (denoted \CTKNLI) is made public due to licensing. Data, tools, and models are available under the \textsf{CC BY-SA 3.0} license.}.
\end{enumerate}

This article is structured as follows: in Section~\ref{sec:related_work}, we give an overview of the related work.
Section~\ref{sec:fevercs} describes our experimental method to localize the \FEN dataset using the \MediaWiki alignment.
We generate the Czech language \FCZ dataset with it and analyze its validity.
In~Section~\ref{sec:ctkfacts}, we introduce the novel \CTK dataset.
We describe its annotation methodology, data cleaning, and postprocessing, as well as analysis of the inter-annotator agreement.
Section~\ref{sec:analysis} analyzes spurious cues for both \FCZ and \CTK.
In Section~\ref{sec:baseline}, we present the baseline models.
Section~\ref{sec:conclusion} concludes with an overall discussion of the results and with remarks for future research.

% \begin{figure}[H]
% \begin{lstlisting}[language=json]
% {
%   "id": 36242,
%   "verifiable": "VERIFIABLE",
%   "label": "REFUTES",
%   "claim": "Mud was made before Matthew McConaughey was born.",
%   "evidence": [
%     [
%       [52443, 62408, "Mud_-LRB-2012_film-RRB-", 1],
%       [52443, 62408, "Matthew_McConaughey", 0]
%     ],
%     [
%       [52443, 62409, "Mud_-LRB-2012_film-RRB-", 0]
%     ]
%   ]
% }
% \end{lstlisting}
%     \caption{Example \FEN \texttt{REFUTES} annotation with two possible evidence sets. \todo{Take example from \FCZ.}}
%     \label{list:fever}
% \end{figure}
%!TEX ROOT=../itat2022.tex
\section{Related work}\label{sec:related_work}
This section describes datasets and models related to the task of automated fact-checking of textual claims.
More general overview of the state-of-the-art can be found in~\cite{zeng2021fcsurvey} or~\cite{murayama2021dataset}.

Emergent~\cite{ferreira2016emergent} dataset is based on news; it contains 300 claims and 2k+ articles, however, it is limited to headlines.
Due to the dataset size, only simple models classifying to three classes (\textit{for}, \textit{against}, and \textit{observing}) are presented.
Described models are fed BoW vectors and feature-engineered attributes.

Wang in~\cite{wang2017liar} presents another dataset of 12k+ claims, working with 5 classes (\textit{pants-fire}, \textit{false}, \textit{barely-true}, \textit{half-true}, \textit{mostly-true}, and \textit{true}).
Each verdict includes a justification. 
However, evidence sources are missing.
The models presented in the paper are claim-only, i.e., they deal with surface-level linguistic cues only.
The author further experiments with speaker-related meta-data.

Fact Extraction and VERification (\FEVER)~\cite{fever2018} is a large dataset of 185k+ claims covering the overall fact-checking pipeline. 
It was based on abstracts of 50k most visited pages of English \Wikipedia.
Authors present complex annotation methodology that involves two stages: the \textit{claim generation} in which annotators firstly create a true \textit{initial claim} supported by a random \Wikipedia source article with context extended by the \textit{dictionary} constructed from pages linked from the source article.
The \textit{initial claim} is further \textit{mutated} by rephrasing, negating and other operations.
The task of the second \textit{claim labeling} stage is to provide the \textit{evidence} as well as give the final verdict: \texttt{SUPPORTS}, \texttt{REFUTES} or \texttt{NEI}, where the latter stands for the \textit{not enough information} label.
Fact Extraction and VERification Over Unstructured and Structured information (\textsc{FEVEROUS})~\cite{aly2021feverous} adds 87k+ claims including evidence based on \Wikipedia table cells.
The size of \FEVER data facilitates modern deep learning NLP methods.
The \FEVER authors host annual workshops involving competitions, with results described in~\cite{fever2018b} and~\cite{thorne2019fever2}.

\textsc{MultiFC}~\cite{augenstein2019multifc} is a 34k+ claim dataset sourcing its claims from 26 fact checking sites.
The evidence documents are retrieved via Google Search API as the ten highest-ranking results.
This approach significantly deviates from the \FEVER-like datasets as the ground-truth is not limited by a closed-world corpus, which limits the trustworthiness of the retrieved evidence.
Also, similar data cannot be utilized to train the DR models.

\textsc{WikiFactCheck-English}~\cite{sathe2020automated} is another recent \Wikipedia-based large dataset of 124k+ claims and further 34k+ ones including claims refuted by the same evidence.
The claims are accompanied by \textit{context}.
The evidence is based on \Wikipedia articles as well as on the linked documents. 

Considering other than English fact-checking datasets, the situation is less favorable.
Recently, Gupta et al.~\cite{gupta2021xfact} released a multilingual (25 languages) dataset of 31k+ claims annotated by seven veracity classes.
Similarly to the \textsc{MultiFC}, evidence is retrieved via Google Search API.
The experiments with the multilingual \BERT~\cite{devlin2019bert} model show that the gain from including the evidence is rather limited when compared to claim-only models.
FakeCovid~\cite{shahi2020fakecovid} is a multilingual (40 languages) dataset of 5k+ news articles.
The dataset focuses strictly on the COVID-19 topic.
Also, it does not supply evidence in a raw form -- human fact-checker argumentation is provided instead. 
Kazemi et al.~\cite{kazemi2021claim} released two multilingual (5 languages) datasets, these are, however, aimed at \textit{claim detection} (5k+ examples) and \textit{claim matching} (2k+ claim pairs).

In the Czech locale, the most significant machine-learnable dataset is the \textsc{Demagog} dataset~\cite{priban-etal-2019-machine} based on the fact-checks of the Demagog\footnote{\url{https://demagog.cz/}} organisation.
The dataset contains 9k+ claims in Czech (and 15k+ in Slovak and Polish) labeled with a veracity verdict and speaker-related metadata, such as name and political affiliation.
The verdict justification is given in natural language, often providing links from social networks, government-operated webpages, etc.
While the metadata is appropriate for statistical analyses, the justification does not come from a closed knowledge base that could be used in an automated scheme.

The work most related to ours was presented by the authors of~\citep{binau2020danish,norregaard2021danfever}, who published a Danish version of \FEN called \FDAN.
Unlike our \FCZ dataset, \FDAN was annotated by humans.
Given the limited number of annotators, it includes significantly fewer claims than \FEN (6k+ as opposed to 185k+).
% , which makes it less appealing for the state-of-the-art neural models.


%\todo{The following paragraphs (commented out) should be moved to their corresponding sections.}

\begin{comment}
\textbf{DR datasets and methods:}
\jd{Relation to QA and QA datasets. The connection: DRQA~\cite{chen2017drqa}~\cite{fever2018b} add MORE}

\textbf{NLI datasets and methods:}
We have examined the following NLI datasets in English, and their respective state-of-the-art classifiers, largerly based on transformer models resemblant to \BERT~\cite{devlin2019bert}.
The original \FEVER paper refers to this task as to \textit{Recognizing Textual Entailment} (RTE)~\cite{fever2018}.
As RTE is mostly understood as a binary classification problem (\texttt{entailed} and \texttt{no entailment} labels)~\cite{chatzikyriakidis2017overview}, we recognize the task as the \textit{Natural Language Inference} (NLI).
NLI is stated as a three-way classifiction with \texttt{entailed}, \texttt{negation entailed} and \texttt{no entailment} labels which directly map to \texttt{SUPPORTS}, \texttt{REFUTES} and \texttt{NEI}.

\textsc{SNLI}~\cite{bowman2015large} (corpus of approx. 570,000 human-written English sentence pairs manually labeled for balanced classification).
\textsc{MultiNLI}~\cite{williams2018broad} (approx. 433,000 sentence pairs and covers various genres of spoken and written English)
\textsc{ANLI}~\cite{nie2020adversarial} (approx. 170,000 sentence pairs in human-and-model-in-the-loop dataset, consisting of three rounds of increasing complexity and difficulty (\textsf{A1}, \textsf{A2}, \textsf{A3}).
\FEVERNLI~\cite{nie2019combining} is a simple conversion of the \FEVER dataset from its original format to the $(query,context)$ pairs.
    
\textbf{Multilingual Language Models:}
\MBERT (\textit{Multilingual \BERT}) is a variation of \BERT$_\textsf{BASE}$ model~\cite{devlin2019bert} (\jd{Is'n it LARGE model?, how many languages?} -- \hu{The one we used in my thesis was \url{https://huggingface.co/DeepPavlov/bert-base-multilingual-cased-sentence} -- BASE. 101 langs}).

\XLM~\cite{conneau2020unsupervised} is the crosslingual version of \textsc{RoBERTa} (\jd{How many languages?}).

\textbf{Slavonic Language Models:} 
\SlavicBERT~\cite{arkhipov2019tuning} is similar to \MBERT, trained on joint Bulgarian, Czech, Polish and Russian corpora.

\CZERT~\cite{sido2021czert} and \RobeCzech~\cite{straka2021robeczech} are recent Czech monolingual models based \BERT and \RoBERTa families.


\todo{Move this to the "Model-based annotation cleaning" section}
This paper~\cite{brod6ley1996identifying} describes our approach to cleaning dataset based on cross-validation. The differences are: we use a single classifier only (instead of ensemble of $m$ classifiers). Moreover, we use this approach to detect the mislabelled samples only. Instead of removing them, we fix the label.
\end{comment}
\begin{comment}
    \section{Discussion}\label{sec:discussion}

    Discussions should be brief and focused...
\end{comment}
\section{Conclusion}\label{sec:conclusion}

With this paper, we address the lack of a Czech dataset for automated fact-checking.
We have explored two ways of acquiring such data.

Firstly, we localize the \FEN dataset, using a document alignment between Czech and English \Wikipedia abstracts extracted from the \textit{interlingual links}.
We obtain and publish the \FCZ dataset of 127k machine-translated claims with evidence enclosed within the Czech \Wikipedia dump.
We then validate our alignment scheme and measure a 66\% precision using hand annotations over a 1\% sample of obtained data.
Therefore, we recommend the data for models less sensitive to noise and utilize it to train experimental DR models and for recall estimation.

Secondly, we executed a series of human annotation runs with 163 students of journalism to acquire a novel dataset in Czech.
As opposed to similar annotations that extracted claims and evidence from \Wikipedia~\cite{fever2018,norregaard2021danfever,aly2021feverous}, we annotated our dataset on top of a CTK corpus extracted from a news agency archive to explore this different relevant language form. 
We collected a raw dataset of 3,116 labeled claims, 57\% of which have at least two independent cross-annotations.
From these, we calculate Krippendorff's alpha to be 56.42\%.
We proceed with manual and human-and-model-in-the-loop annotation cleaning to remove conflicting and malformed annotations, arriving at the thoroughly cleaned \CTK dataset of 3,097 claims and their veracity annotations complemented with evidence from the CTK corpus.
We release its version for NLI called \CTKNLI to maintain corpus trade secrecy.

Finally, we use our datasets to train baseline models for the full fact-checking pipeline composed of Document Retrieval and Natural Language Inference tasks.
% At five retrieved paragraphs in a Score Evidence setting, where at least one gold set of evidence must be covered by the DR predictions, our best performing models scored 33.89\% overall accuracy on \FCZ and 20.14\% on \CTK.
% In a No Score Evidence setting, where the gold evidence requirement is removed, we scored 55.40\% and 61.24\% on \FCZ and \CTK, respectively.
% We claim the results to be testifying to the viability of the task in our setting and encouraging for further research on our data.

\subsection{Future work}
\begin{itemize}
    \item The fact-checking pipeline is to be augmented by the \textit{check-worthiness estimation}~\cite{nakov2021automated}, that is, a model that classifies which sentences of a given text in Czech are appropriate for the fact verification.
    We are currently working on models that detect claims within the Czech Twitter, and a strong predictor for this task would also strengthen our annotation scheme from Section~\ref{sec:annotation} that currently relies on hand-picked check-worthy documents.
    \item While the \SUP, \REF{} and \NEI{} classes  offer a finer classification w.r.t. evidence than binary \texttt{true}/\texttt{false}, it is a good convention of fact-checking services to use additional labels such as \texttt{MISINTERPRETED}, that could be integrated into the common automated fact verification scheme if well formalised.
    \item The claim extraction schemes like that from~\cite{fever2018} or Section~\ref{sec:annotation} do not necessarily produce organic claims capturing the real-world complexity of fact-checking.
    For example, just the \FEN \train set contains hundreds of claims of form \enquote{X is a person.}.
    This problem does not have a trivial solution, but we suggest integrating real-world claims sources, such as Twitter, into the annotation scheme.
    \item While the \FEVER localization scheme from Section~\ref{fcz-method} yielded a rather noisy dataset, its size and document precision encourage
    deployment of a model-based cleaning scheme like that from~\cite{Jeatrakul} to further refine its results.
\end{itemize}

\backmatter

% \bmhead{Supplementary information}

% If your article has accompanying supplementary file/s please state so here. 

%\bmhead{Acknowledgments}

%%===========================================================================================%%
%% If you are submitting to one of the Nature Portfolio journals, using the eJP submission   %%
%% system, please include the references within the manuscript file itself. You may do this  %%
%% by copying the reference list from your .bbl file, paste it into the main manuscript .tex %%
%% file, and delete the associated \verb+\bibliography+ commands.                            %%
%%===========================================================================================%%

\bibliography{itat2022}
\pagebreak

\begin{appendices}
\end{appendices}

\end{document}
